% xelatex
\documentclass[12pt, twoside, final]{ruost}

% Словарь
\usepackage{polyglossia}   %% загружает пакет многоязыковой вёрстки
\setdefaultlanguage[spelling=modern]{russian}  %% главный язык документа
\setotherlanguage{english} %% второй язык документа

% Шрифты документа
\defaultfontfeatures{Ligatures={TeX},Renderer=Basic}
\setmainfont{Arial}

% Настройка математики
\usepackage{amsmath}
\usepackage[math-style=upright]{unicode-math}
% Загрузка математического шрифта
\setmathfont{Cambria Math}

% Подмена основных символов (цифры, латиница, греческие буквы)
\setmathfont[range=\mathup/{num,latin,Latin,greek,Greek}]{Arial}
\setmathfont[range=\mathit/{num,latin,Latin,greek,Greek}]{Arial Italic}

% Подмена других символов
% Символы:            ,     .     :     ;
\setmathfont[range={"002C,"002E,"003A,"003B}]{Arial}
\setmathfont[range=\sum, Scale=1.5]{Arial}

% Возврат надстрочных и подстрочных индлексов из математического шрифта
% Иначе возможны проблемы со степенными корнями
\setmathfont[range={"2070-"208F}]{Cambria Math}


% Расстановка отступов и переносов
\emergencystretch=2em
\tolerance=1000
\frenchspacing

\usepackage[unicode, hidelinks, hypertexnames=false]{hyperref}
\usepackage{float}

\usepackage{longtable}
\usepackage{tabu}
\renewcommand{\arraystretch}{1.3}
\captionsetup[longtable]{margin=1.5pt} % Хак для выравнивания заголовков длинных таблиц
\setlength{\LTpre}{\parskip}
\setlength{\LTpost}{\parskip}

\usepackage{metalogo} % XeLaTeX logo


\renewcommand{\OSTTitleLogo}{{\Huge\bfseries \fbox{RuOST}}}
\renewcommand{\OSTOrganisation}{Наименование Организации}
\renewcommand{\OSTCity}{Город}

\renewcommand{\OSTApproverState}{Генеральный директор}
\renewcommand{\OSTApproverName}{Иванов И.И.}

\renewcommand{\OSTHeader}{Класс \LaTeX{} для стандартов организаций}
\renewcommand{\OSTSubheader}{RuOST 1.0}
\renewcommand{\OSTNumber}{RuOST 1.0}
\renewcommand{\OSTApplyDate}{ГГГГ-ММ-ДД}

\renewcommand{\OSTLastpageClassLeft}{Код УДК}
\renewcommand{\OSTLastpageClassCenter}{Код ОКС}
\renewcommand{\OSTLastpageClassRight}{Код ОКП}
\renewcommand{\OSTLastpageKeywords}{ключевые, слова, стандарта}

\newcommand{\txtcmd}[1]{\textbf{\backslash{}#1}}

\begin{document}

	\maketitle
	
	\OSTPresection{Предисловие}
	
		Общие сведения о стандарте.
	
		\subsection*{Сведения о стандарте}
	
		\par 1\OSTHspace{}РАЗРАБОТАН
		\par 2\OSTHspace{}УТВЕРЖДЁН И ВВЕДЁН В ДЕЙСТВИЕ
		\par 3\OSTHspace{}ВВЕДЕН ВПЕРВЫЕ
		
		\vfill
		\begin{flushright}
			\copyright{} Организация, 2014
		\end{flushright}
		
		Настоящий стандарт не может быть полностью или частично воспроизведен, тиражирован и
		распространен в качестве официального издания без разрешения Организации.
	
	\newpage
	\tableofcontents
	\newpage
	
	\begin{OST}
		
		\section{Использование класса}
				
			\subsection{Параметры загрузки класса}\label{sec:args}
			
				\point Размер основного шрифта документа определяется параметрами класса  \textbf{10pt}, \textbf{12pt}, \textbf{14pt}. Значение по умолчанию \textbf{12pt}.
				\point Для получения отметки <<Издание официальное>> вместо <<Проект стандарта>> и <<Проект>> на титульном и первом листах стандарта необходимо использовать параметр класса \textbf{final}.
				\point Параметр \textbf{twoside} включает режим двусторонней печати.
				\point Для включения нумерации объектов в пределах разделов документа необходимо использовать параметры класса \textbf{figuresection}, \textbf{tablesection}, \textbf{equationsection}.
					\subpoint Требования к нумерации объектов изложены в разделах \ref{sec:tables}, \ref{sec:figures}, \ref{sec:equations}.
				\point Для включения нумерации приложений арабскими цифрами необходимо использовать параметр класса \textbf{arabicappendixes}.
					\subpoint Требования к нумерации приложений изложены в пункте \ref{pnt:numappendixes}.
				\point Для включения нумерации сносок арабскими цифрами необходимо использовать параметр класса \textbf{arabicfootnotes}.
				\subpoint Требования к нумерации сносок изложены в пункте \ref{pnt:numfootnotes}.
				
			
			\subsection{Дополнительные команды}\label{sec:commands}
			
				\point Для заголовков разделов идущих до основного текста стандарта (Предисловие, Введение и т.п.) существует команда \txtcmd{OSTpresection\{Заголовок\}}.
				\point Для приложений существует команда \txtcmd{OSTappendix\{Тип приложения\} \{Заголовок\}}.
				\subpoint Как правило, тип приложения это <<(обязательное)>>, <<(рекомендуемое)>> или <<(справочное)>>.
				\note{Приложения должны быть расположены после всех разделов текста стандарта, иначе возможны проблемы с нумерацией объектов.}
				\point Текст стандарта (кроме предварительных разделов и титульного листа) необходимо располагать внутри окружения \textbf{OST}, которое переключает нумерацию на арабские цифры и формирует первую и последнюю страницы основного текста стандарта.
				\point Пункты и подпункты формируются командами \txtcmd{point} и \txtcmd{subpoint}.
				\point Для создания одиночного примечания необходимо использовать команду \txtcmd{note\{Текст\}}.
				\point Для перечисления примечаний необходимо использовать окружение \textbf{notes}, внутри которого каждое примечание должны быть расположено после команды \txtcmd{item}.
				\point Для повторения положений другого стандарта необходимо использовать окружение \textbf{stdquote}.
				
				\point Кроме того, в классе переопределены и типовые команды: \txtcmd{maketitle}, \txtcmd{tableofcontents}, окружение \textbf{thebibliography}.
			
			\subsection{Основные переопределяемые команды}
			
				\point Для заполнения титульного, первого и последнего листов стандарта необходимо переопределить следующие команды.
					\subpoint \txtcmd{OSTOrganisation} --- название компании, изображается в верхней части титульного листа.
					\subpoint \txtcmd{OSTTitleLogo} --- логотип компании, изображается в левой верхней части титульного листа под названием компании.
					\subpoint \txtcmd{OSTCity} --- город, где подготовлен стандарт, отображается в нижней части титульного листа.
					\subpoint \txtcmd{OSTApproverState} --- должность лица утверждающего, обычно директор компании. Отображается на титульном листе в блоке <<Утверждаю>>.
					\subpoint \txtcmd{OSTApproverName} --- фамилия и инициалы лица утверждающего. Отображается на титульном листе в блоке <<Утверждаю>>.
					\subpoint \txtcmd{OSTHeader} --- заголовок стандарта, отображается на титульном и первом листах стандарта.
					\subpoint \txtcmd{OSTSubheader} --- подзаголовок стандарта, отображается на титульном и первом листах стандарта.
					\subpoint \txtcmd{OSTNumber} --- часть номера стандарта в которую входит код ОКПО организации и внутренний регистрационный номер стандарта.
					\subpoint \txtcmd{OSTApplyDate} --- дата принятия стандарта (ГГГГ-ММ-ДД), отображается на первом листе стандарта.
					\subpoint \txtcmd{OSTLastpageClassLeft} --- классификационный код отображаемый в левой верхней части последней страницы стандарта.
					\subpoint \txtcmd{OSTLastpageClassCenter} --- классификационный код отображаемый в центральной верхней части последней страницы стандарта.
					\subpoint \txtcmd{OSTLastpageClassRight} --- классификационный код отображаемый в правой верхней части последней страницы стандарта.
					\subpoint \txtcmd{OSTLastpageKeywords} --- ключевые слова стандарта, отображаются на последней странице.
				\point По умолчанию в незаполненные поля выводятся имена команд без обратного слэша.
			
			\subsection{Дополнительные переопределяемые команды}
			
				\point Горизонтальный отступ после номера пункта, раздела или подраздела определяется командой \txtcmd{OSTSep}, по умолчанию равен \textbf{1ex}.
				
				\point Абзацный отступ определяется командой \txtcmd{OSTParindent}, по умолчанию равен \textbf{5ex}.
				
					\begin{stdquote}
						\par 6.1.3 Абзацный отступ должен быть одинаковым по всему тексту проекта стандарта и равен пяти знакам.
						\par [ГОСТ Р 1.5---2012]
					\end{stdquote}
					
				\note{Если необходимо изменить абзацный отступ, переопределяйте именно \txtcmd{OSTParindent}, т.к. он этой длины зависит абзацный отступ не только в основном тексте документа, но и внутри некоторых окружений.}
				
				\point Стили шрифта заголовков определяются следующими командами.
				
					\subpoint \txtcmd{OSTFontHeader} --- стиль шрифта заголовка стандарта, по умолчанию команда определена как
					\txtcmd{Large\backslash{}bfseries\backslash{}MakeUppercase}.
					
					\subpoint \txtcmd{OSTFontSubheader} --- стиль шрифта подзаголовка стандарта, по умолчанию команда определена как \txtcmd{Large\backslash{}bfseries}.
					
					\subpoint \txtcmd{OSTFontSection} --- стиль шрифта заголовка раздела стандарта, по умолчанию команда определена как \txtcmd{large\backslash{}bfseries}.
					
					\subpoint \txtcmd{OSTFontSubsection} --- стиль шрифта заголовка подраздела, по умолчанию стандарта команда определена как \txtcmd{normalsize\backslash{}bfseries}.
					
				\point Отступы перед и после заголовков определены следующими командами.
				
					\subpoint \txtcmd{OSTBeforeSectionSkip} --- пропуск перед заголовком раздела, по умолчанию \txtcmd{baselineskip}
					
					\subpoint \txtcmd{OSTAfterSectionSkip} --- пропуск после заголовка раздела, по умолчанию \txtcmd{baselineskip}
					
					\subpoint \txtcmd{OSTBeforeSubsectionSkip} --- пропуск перед заголовком подраздела, по умолчанию \textbf{.5\backslash{}baselineskip}
					
					\subpoint \txtcmd{OSTAfterSubsectionSkip} --- пропуск после заголовка подраздела, по умолчанию \textbf{.5\backslash{}baselineskip}

				
				\point Текстовые константы, определяющие тип стандарта.
				
					\subpoint \txtcmd{OSTDocumentClassI} --- тип стандарта, отображается на титульном листе.
					
					\subpoint \txtcmd{OSTDocumentClassII} --- тип стандарта, отображается на первом листе.
					
					\subpoint \txtcmd{OSTClassPrefix} --- префикс перед номером стандарта,  по умолчанию команда определена как \textbf{СТО}.
					
				\point  \txtcmd{OSTYear} --- год разработки стандарта, отображается в нижней части титульного листа,  по умолчанию команда определена как \txtcmd{the\backslash{}year}.
		
		\section{Оформление стандарта организации}
		
			\subsection{Документы регламентирующие оформление}
			
				\point Стандарт организации оформляется по ГОСТ~Р~1.5.
				
					\begin{stdquote}
						\par 4.12 Построение, изложение, оформление и содержание стандартов организаций выполняются с учетом требований ГОСТ~Р~1.5.
						\par [ГОСТ Р 1.4---2004]
					\end{stdquote}
				
				\point В свою очередь ГОСТ~Р~1.5, в правилах оформления стандарта, за исключением особенностей для национальных стандартов, полностью ссылается на ГОСТ~1.5.
				
					\begin{stdquote}
						\par 3.1 Национальный стандарт Российской Федерации и предварительный национальный стандарт (далее — стандарт, за исключением случаев, когда необходимо конкретизировать статус стандарта) состоят из отдельных элементов, состав которых установлен ГОСТ~1.5—2001 (подраздел 3.1).
						\par [ГОСТ Р 1.5---2012]
					\end{stdquote}
					
					\begin{stdquote}
						\par 3.4 При включении в стандарт дополнительных элементов «Содержание» и «Введение» применяют правила, установленные ГОСТ~1.5—2001 (подразделы 3.4 и 3.5).
						\par [ГОСТ Р 1.5---2012]
					\end{stdquote}
					
					\begin{stdquote}
						\par 3.7.1 Элемент «Термины и определения» излагают и оформляют с соблюдением правил, установленных ГОСТ~1.5—2001 (подраздел 3.9).
						\par [ГОСТ Р 1.5---2012]
					\end{stdquote}
					
					\begin{stdquote}
						\par 3.7.2 При необходимости в стандарте допускается повторять определение термина, которое установлено в стандарте на термины и определения, действующем в Российской Федерации на национальном уровне. При этом при оформлении терминологических статей соблюдают правила, установленные ГОСТ~1.5—2001 (пункт 4.8.4).
						\par [ГОСТ Р 1.5---2012]
					\end{stdquote}
					
					\begin{stdquote}
						\par 3.8 Если в стандарте необходимо использовать значительное число (более пяти) обозначений и/или сокращений, то для их установления используют один из следующих элементов стандарта: «Обозначения и сокращения», «Обозначения», «Сокращения», который оформляют по правилам, установленным в ГОСТ~1.5—2001 (подраздел 3.10).
						\par [ГОСТ Р 1.5---2012]
					\end{stdquote}
					
					\begin{stdquote}
						\par 3.9 Нормативные положения основной части стандарта оформляют в виде разделов, состав и содержание которых устанавливают с учетом особенностей объекта и аспекта стандартизации, а также общих требований к содержанию стандартов, установленных ГОСТ~1.5—2001 (раздел 7) для данного вида стандарта, и требований, установленных в отношении отдельных объектов и аспектов стандартизации ГОСТ~Р~51898, ГОСТ~Р~14.08, ГОСТ~Р~ИСО/МЭК~50, ГОСТ~Р~54930 и ГОСТ~Р~54937.
						\par [ГОСТ Р 1.5---2012]
					\end{stdquote}
					
					\begin{stdquote}
						\par 3.10 Материал, дополняющий нормативные положения основной части стандарта, оформляют в виде приложений с соблюдением правил, установленных ГОСТ~1.5—2001 (подраздел 3.12).
						\par [ГОСТ Р 1.5---2012]
					\end{stdquote}
					
					\begin{stdquote}
						\par 3.11 Если в стандарте даны справочные ссылки в соответствии с 4.4, то в данный стандарт включают дополнительный элемент «Библиография» с соблюдением правил, установленных ГОСТ~1.5—2001 (подраздел 3.13).
						\par [ГОСТ Р 1.5---2012]
					\end{stdquote}
					
					\begin{stdquote}
						\par 3.13 При необходимости несколько стандартов могут быть сброшюрованы в тематический сборник в соответствии с правилами, установленными ГОСТ~1.5—2001 (пункт 3.2.3).
						\par [ГОСТ Р 1.5---2012]
					\end{stdquote}
					
					\begin{stdquote}
						\par 4.1 При изложении стандарта применяют соответствующие положения ГОСТ~1.5—2001 (раздел 4) с дополнениями, приведенными в настоящем разделе.
						\par [ГОСТ Р 1.5---2012]
					\end{stdquote}
					
					\begin{stdquote}
						\par 4.4.3 Информацию о ссылочных документах, на которые даны справочные ссылки, приводят в дополнительном элементе «Библиография», который оформляют по ГОСТ~1.5—2001 (подраздел 3.13).
						\par [ГОСТ Р 1.5---2012]
					\end{stdquote}
					
					\begin{stdquote}
						\par 4.6 В случае, когда в стандарте целесообразно повторить какое-либо положение другого национального стандарта Российской Федерации (действующего в этом качестве межгосударственного стандарта или свода правил), то применяют соответствующее правило, установленное ГОСТ~1.5—2001 (пункт 4.8.4).
						\par [ГОСТ Р 1.5---2012]
					\end{stdquote}
					
					\begin{stdquote}
						\par 4.7 В стандарте могут быть использованы условные обозначения, изображения и знаки, которые установлены в других национальных стандартах Российской Федерации и действующих в этом качестве межгосударственных стандартах. Если условные обозначения, изображения или знаки не установлены в таких стандартах, то эти условные обозначения, изображения, знаки поясняют в тексте или в разделе «Обозначения и сокращения», который оформляют в соответствии с ГОСТ~1.5—2001 (подраздел 3.10).
						\par [ГОСТ Р 1.5---2012]
					\end{stdquote}
					
					\begin{stdquote}
						\par 5.1 При оформлении проекта стандарта и при подготовке к опубликованию утвержденного стандарта применяют правила, установленные ГОСТ~1.5—2001 (раздел 6) с дополнениями, приведенными в настоящем разделе.
						\par [ГОСТ Р 1.5---2012]
					\end{stdquote}
					
				\point Таким образом, для подготовки стандарта организации необходимо изучить требования ГОСТ~1.5, затем требования ГОСТ~Р~1.5 и, наконец, ГОСТ~Р~1.4.
				
			\subsection{Деление текста}
				
				\point Текст стандарта разбивают на разделы, подразделы, пункты и подпункты.
				
					\begin{stdquote}
						\par 4.2.1 Текст основной части стандарта делят на структурные элементы: разделы, подразделы, пункты, подпункты. Разделы могут делиться на пункты или на подразделы с соответствующими пунктами. Пункты при необходимости могут делиться на подпункты.
						\par [ГОСТ 1.5---2001]
					\end{stdquote}
				
				\point Пункты и подпункты формируются командами \txtcmd{point} и \txtcmd{subpoint}.
				\point Нумерация разделов и пунктов выполняется арабскими цифрами, точку после номеров не ставят.
				
					\begin{stdquote}
						\par 4.2.1.2 Разделы, подразделы, пункты и подпункты нумеруют арабскими цифрами.
						\par [ГОСТ 1.5---2001]
					\end{stdquote}
					
					\begin{stdquote}
						\par 4.2.1.7 После номера раздела, подраздела, пункта и подпункта точку не ставят, а отделяют от текста стандарта пробелом.
						\par [ГОСТ 1.5---2001]
					\end{stdquote}
					
				\point\label{pnt:appendixsections} Разделы и подразделы могут быть и в пределах приложения.
				
					\begin{stdquote}
						\par 4.2.2 Текст приложения может быть разделен на разделы, подразделы, пункты, подпункты, которые нумеруют в пределах каждого приложения, ставя перед их номерами обозначение этого приложения и отделяя его от номера точкой.
						\par [ГОСТ 1.5---2001]
					\end{stdquote}
				
				\point\label{pnt:numappendixes} Приложения нумеруются буквами русского алфавита с некоторыми исключениями либо цифрами.
				
					\begin{stdquote}
						\par 3.12.3 Приложения обозначают прописными буквами русского алфавита, начиная с А (за исключением букв Ё, 3, И, О, Ч, Ь, Ы, Ъ), которые приводят после слова «Приложение».
						\par В случае полного использования букв русского алфавита приложения обозначают арабскими цифрами.
						\par [ГОСТ 1.5---2001]
					\end{stdquote}
					
					\note{Для исключения путаницы с номерами разделов при использовании параметра класса включающего цифровую нумерацию приложений (\textbf{arabicappendixes}) перед их номером выводится префикс (по умолчанию буква <<П>>), определяемый командой \txtcmd{OSTAppendixNumericPrefix}.}
					
			\subsection{Заголовки}
			
				\point Регламентированы следующие требования к оформлению заголовков.
				
					\begin{stdquote}
						\par 4.3.3 Заголовок раздела (подраздела или пункта) печатают, отделяя от номера пробелом, начиная с прописной буквы, не приводя точку в конце и не подчеркивая. При этом номер раздела (подраздела или пункта) печатают после абзацного отступа, оформляемого в соответствии с 6.1.3.
						\par [ГОСТ 1.5---2001]
					\end{stdquote}
					
					\begin{stdquote}
						\par 4.3.6 В стандарте заголовки разделов, подразделов, пунктов выделяют полужирным шрифтом. При этом заголовки разделов (а при наличии заголовков пунктов также заголовки подразделов) выделяют увеличенным размером шрифта.
						\par [ГОСТ 1.5---2001]
					\end{stdquote}
			
				\point Регламентированы и вертикальные пропуски перед и после заголовков раздела и подраздела.
				
					\begin{stdquote}
						\par 6.1.2 Расстояние между заголовком раздела (подраздела) и предыдущим или последующим текстом, а также между заголовками раздела и подраздела должно быть равно не менее чем четырем высотам шрифта, которым набран основной текст стандарта.
						\par [ГОСТ 1.5---2001]
					\end{stdquote}
					\note{Это требование является довольно странным (т.к. требует слишком большие пропуски), и ему не удовлетворяет даже устанавливающий это требование ГОСТ~1.5---2001 (в тексте самого стандарта они существенно меньше). В данном классе используются значения подобранные для максимального визуального сходства с оформлением самого ГОСТ~1.5---2001. Пропуски определяются командами: \txtcmd{OSTBeforeSectionSkip}, \txtcmd{OSTAfterSectionSkip}, \txtcmd{OSTBeforeSubsectionSkip}, \txtcmd{OSTAfterSubsectionSkip}.}
				
			
			\subsection{Перечисления}
			
				\point Регламентированы следующие требования к оформлению перечислений.
				
					\begin{stdquote}
						\par 4.4.2 Перечисления выделяют в тексте абзацным отступом, который используют только в первой строке.
						\par 4.4.3 Перед каждой позицией перечисления ставят дефис.
						\par 4.4.4 Если необходимо в тексте стандарта сослаться на одно или несколько перечислений, то перед каждой позицией вместо дефиса ставят строчную букву, приводимую в алфавитном порядке, а после нее - скобку.
						\par 4.4.5 Для дальнейшей детализации перечисления используют арабские цифры, после которых ставят скобку, приводя их со смещением вправо на два знака относительно перечислений, обозначенных буквами.
						\par [ГОСТ 1.5---2001]
					\end{stdquote}
					\begin{notes}
						\item Не смотря на то, что эти требования не полностью регламентируют оформление перечислений, приведенного в стандарте примера достаточно для определения остальных параметров.
						\item Стандарт не определяет недопустимых в нумерованных перечислениях букв. В данном классе используются те же ограничения для нумерации перечислений, что и для приложений.
					\end{notes}
				
				\point Стандарт регламентирует стиль только двух уровней вложенности нумерованных списков.
				
				\begin{enumerate}
					\item нумерованный элемент перечисления;
					\item нумерованный элемент перечисления;
					\item нумерованный элемент перечисления, занимающий для большей наглядности более одной строки и показывающий левую границу текста на новых строках;
					\begin{enumerate}
						\item нумерованный элемент перечисления;
						\item нумерованный элемент перечисления;
						\item нумерованный элемент перечисления, занимающий для большей наглядности более одной строки и показывающий левую границу текста на новых строках.
					\end{enumerate}
				\end{enumerate}
				
				\point В данном классе в качестве маркера ненумерованных списков используется удлиненный дефис (en-dash), определяемый командой \txtcmd{OSTListDash}.
				
				\point Стандарт не регламентирует максимальный уровень вложенности списков, однако данный класс поддерживает не более четырех.
				
				\begin{itemize}
					\item ненумерованный элемент перечисления;
					\item ненумерованный элемент перечисления;
					\item ненумерованный элемент перечисления, занимающий для большей наглядности более одной строки и показывающий левую границу текста на новых строках;
					\begin{itemize}
						\item ненумерованный элемент перечисления;
						\item ненумерованный элемент перечисления;
						\item ненумерованный элемент перечисления, занимающий для большей наглядности более одной строки и показывающий левую границу текста на новых строках;
						\begin{itemize}
							\item ненумерованный элемент перечисления;
							\item ненумерованный элемент перечисления;
							\item ненумерованный элемент перечисления, занимающий для большей наглядности более одной строки и показывающий левую границу текста на новых строках;
							\begin{itemize}
								\item ненумерованный элемент перечисления;
								\item ненумерованный элемент перечисления;
								\item ненумерованный элемент перечисления, занимающий для большей наглядности более одной строки и показывающий левую границу текста на новых строках.
							\end{itemize}
						\end{itemize}
					\end{itemize}
				\end{itemize}
				
				\point Расстояние от начала маркера пункта списка до начала текста пункта определяется значение \txtcmd{OSTListLabelWidth}.
				
				\point Судя по примеру в тексте ГОСТ~1.5---2001, смещение в 2 символа на каждом последующем уровне вложенности задаётся от начала текста пункта верхнего уровня, до начала метки пункта нижнего уровня. Это смещение определено командой \txtcmd{OSTListIndent}.
				
			\subsection{Таблицы}\label{sec:tables}
			
				\point Регламентированы следующие требования к оформлению таблиц.
				
					\begin{stdquote}
						\par Слева над таблицей размещают слово «Таблица», выделенное разрядкой. После него приводят номер таблицы, присваиваемый в соответствии с 4.5.3. При этом точку после номера таблицы не ставят.
						\par При необходимости краткого пояснения и/или уточнения содержания таблицы приводят ее наименование, которое записывают с прописной буквы над таблицей после ее номера, отделяя от него тире. При этом точку после наименования таблицы не ставят.
						\par Горизонтальные линии, разграничивающие строки таблицы, допускается не проводить, если их отсутствие не затрудняет пользование таблицей.
						\par [ГОСТ 1.5---2001, пункт 4.5.2]
					\end{stdquote}
					
					\begin{stdquote}
						\par 4.5.3 Таблицы нумеруют арабскими цифрами сквозной нумерацией в пределах всего текста стандарта, за исключением таблиц приложений.
						\par Таблицы каждого приложения нумеруют арабскими цифрами отдельной нумерацией, добавляя перед каждым номером обозначение данного приложения и разделяя их точкой.
						\par Если в стандарте одна таблица, то ее обозначают «Таблица 1» или, например, «Таблица В.1» (если таблица приведена в приложении В).
						\par Допускается нумеровать таблицы в пределах раздела. В этом случае номер таблицы состоит из номера раздела и порядкового номера таблицы, разделенных точкой.
						\par [ГОСТ 1.5---2001]
					\end{stdquote}
					
					\begin{stdquote}
						\par 4.5.6.1 При делении таблицы на части слово «Таблица», ее номер и наименование помещают только над первой частью таблицы, а над другими частями приводят выделенные курсивом слова: «Продолжение таблицы» или «Окончание таблицы» с указанием номера таблицы в соответствии с рисунком 2.
						\par [ГОСТ 1.5---2001]
					\end{stdquote}
					
					\point Для оформления таблиц в документе использован пакет \textbf{tabu}.
					
					\begin{table}[H]
						\caption{Небольшая таблица}\label{tab:st}
						\begin{tabu}{|X[r]|X[c]|X[l]|}
							\hline
							Это    & небольшая & таблица \\ \hline
							В ней  & всего две & строки  \\ \hline
						\end{tabu}
					\end{table}
					
					\begin{longtabu}{|X[l]|X[c]|X[c]|X[r]|}
						\caption{Длинная таблица, занимающая несколько страниц}\label{tab:lt} \\ 
						\hline
						\textbf{А} & \textbf{Б} & \textbf{В} & \textbf{Г} \\ \hline
						\endfirsthead
						
						\multicolumn{4}{@{}l}{\itshape Продолжение таблицы \ref{tab:lt}} \\ \hline
						\textbf{А} & \textbf{Б} & \textbf{В} & \textbf{Г} \\ \hline
						\endhead
						
						Пример & очень & длинной & таблицы \\ \hline
						Пример & очень & длинной & таблицы \\ \hline
						Пример & очень & длинной & таблицы \\ \hline
						Пример & очень & длинной & таблицы \\ \hline
						Пример & очень & длинной & таблицы \\ \hline
						
						Пример & очень & длинной & таблицы \\ \hline
						Пример & очень & длинной & таблицы \\ \hline
						Пример & очень & длинной & таблицы \\ \hline
						Пример & очень & длинной & таблицы \\ \hline
						Пример & очень & длинной & таблицы \\ \hline
						
						Пример & очень & длинной & таблицы \\ \hline
						Пример & очень & длинной & таблицы \\ \hline
						Пример & очень & длинной & таблицы \\ \hline
						Пример & очень & длинной & таблицы \\ \hline
						Пример & очень & длинной & таблицы \\ \hline
						
						Пример & очень & длинной & таблицы \\ \hline
						Пример & очень & длинной & таблицы \\ \hline
						Пример & очень & длинной & таблицы \\ \hline
						Пример & очень & длинной & таблицы \\ \hline
						Пример & очень & длинной & таблицы \\ \hline
						
						Пример & очень & длинной & таблицы \\ \hline
						Пример & очень & длинной & таблицы \\ \hline
						Пример & очень & длинной & таблицы \\ \hline
						Пример & очень & длинной & таблицы \\ \hline
						Пример & очень & длинной & таблицы \\ \hline
						
						Пример & очень & длинной & таблицы \\ \hline
						Пример & очень & длинной & таблицы \\ \hline
						Пример & очень & длинной & таблицы \\ \hline
						Пример & очень & длинной & таблицы \\ \hline
						Пример & очень & длинной & таблицы \\ \hline
						
						Пример & очень & длинной & таблицы \\ \hline
						Пример & очень & длинной & таблицы \\ \hline
						Пример & очень & длинной & таблицы \\ \hline
						Пример & очень & длинной & таблицы \\ \hline
						Пример & очень & длинной & таблицы \\ \hline
						
						Пример & очень & длинной & таблицы \\ \hline
						Пример & очень & длинной & таблицы \\ \hline
						Пример & очень & длинной & таблицы \\ \hline
						Пример & очень & длинной & таблицы \\ \hline
						Пример & очень & длинной & таблицы \\ \hline
						
						Пример & очень & длинной & таблицы \\ \hline
						Пример & очень & длинной & таблицы \\ \hline
						Пример & очень & длинной & таблицы \\ \hline
						Пример & очень & длинной & таблицы \\ \hline
						Пример & очень & длинной & таблицы \\ \hline
						
						Пример & очень & длинной & таблицы \\ \hline
						Пример & очень & длинной & таблицы \\ \hline
						Пример & очень & длинной & таблицы \\ \hline
						Пример & очень & длинной & таблицы \\ \hline
						Пример & очень & длинной & таблицы \\ \hline
						
						Пример & очень & длинной & таблицы \\ \hline
						Пример & очень & длинной & таблицы \\ \hline
						Пример & очень & длинной & таблицы \\ \hline
						Пример & очень & длинной & таблицы \\ \hline
						Пример & очень & длинной & таблицы \\ \hline
						
						Пример & очень & длинной & таблицы \\ \hline
						Пример & очень & длинной & таблицы \\ \hline
						Пример & очень & длинной & таблицы \\ \hline
						Пример & очень & длинной & таблицы \\ \hline
						Пример & очень & длинной & таблицы \\ \hline
						
						Пример & очень & длинной & таблицы \\ \hline
						Пример & очень & длинной & таблицы \\ \hline
						Пример & очень & длинной & таблицы \\ \hline
						Пример & очень & длинной & таблицы \\ \hline
						Пример & очень & длинной & таблицы \\ \hline
						
						Пример & очень & длинной & таблицы \\ \hline
						Пример & очень & длинной & таблицы \\ \hline
						Пример & очень & длинной & таблицы \\ \hline
						Пример & очень & длинной & таблицы \\ \hline
						Пример & очень & длинной & таблицы \\ \hline
						
						Пример & очень & длинной & таблицы \\ \hline
						Пример & очень & длинной & таблицы \\ \hline
						Пример & очень & длинной & таблицы \\ \hline
						Пример & очень & длинной & таблицы \\ \hline
						Пример & очень & длинной & таблицы \\ \hline
						
					\end{longtabu}
				
			\subsection{Графический материал}\label{sec:figures}
			
				\point Регламентированы следующие требования к оформлению рисунков.
				
				\begin{stdquote}
					\par 4.6.3 Любой графический материал (чертеж, схема, диаграмма, рисунок и т.п.) обозначают в стандарте словом «Рисунок».
					\par 4.6.4 Графический материал, за исключением графического материала приложений, нумеруют арабскими цифрами, как правило, сквозной нумерацией, приводя эти номера после слова «Рисунок». Если рисунок один, то его обозначают «Рисунок 1».
					\par Допускается нумерация графического материала в пределах раздела. В этом случае номер рисунка состоит из номера раздела и порядкового номера рисунка, которые разделяют точкой. 
					\par Графический материал каждого приложения нумеруют арабскими цифрами отдельной нумерацией, добавляя перед каждым номером обозначение данного приложения и разделяя их точкой.
					\par 4.6.5 Слово «Рисунок» и его номер приводят под графическим материалом. Далее может быть приведено его тематическое наименование, отделенное тире.
					\par [ГОСТ 1.5---2001]
				\end{stdquote}
			
				\begin{figure}[H]
					\centering
					\LARGE\TeX\quad\LaTeX\quad\LaTeXe\\
					\caption{Логотипы TeX, LaTeX и LaTeX2e}
				\end{figure}
				
				\begin{figure}[H]
					\centering
					\LARGE\XeLaTeX
					\caption{Логотип XeLaTeX}
				\end{figure}
				
			\subsection{Формулы}\label{sec:equations}
			
				\point Регламентированы следующие требования к оформлению формул.
				
				\begin{stdquote}
					\par 4.7.1 При необходимости в тексте стандарта, таблицах и данных, поясняющих графический материал, могут быть использованы формулы.
					\par 4.7.2 Формулы, за исключением помещаемых в приложениях, таблицах и поясняющих данных к графическому материалу, нумеруют сквозной нумерацией арабскими цифрами. При этом номер формулы записывают в круглых скобках на одном уровне с ней справа от формулы. Если в тексте стандарта приведена одна формула, ее обозначают (1).
					\par Допускается нумерация формул в пределах раздела. В этом случае номер формулы состоит из номера раздела и порядкового номера формулы, разделенных точкой.
					\par 4.7.3 Формулы, помещаемые в приложениях, нумеруют арабскими цифрами отдельной нумерацией в пределах каждого приложения, добавляя перед каждым номером обозначение данного приложения и разделяя их точкой.
					\par [ГОСТ 1.5---2001]
				\end{stdquote}
				
				\point В регламентирующих стандартах требований к шрифтам готового документа не содержится, но содержатся требования к шрифтам проекта стандарта.				
				
				\begin{stdquote}
					\par 6.1.1 Проект стандарта оформляют машинным способом. При этом используют гарнитуру шрифта Arial и Symbol, а также соблюдают требования к редактируемым и нередактируемым электронным форматам документов, которые установлены в правилах по порядку обмена документами в электронном формате [3].
					\par [ГОСТ 1.5---2001]
				\end{stdquote}
				
				\begin{stdquote}
					\par 5.2 При оформлении проекта стандарта допускается использование гарнитуры шрифта Times New Roman размером 14 для основного текста и размером 12 для приложений, примечаний, сносок и примеров.
					\par [ГОСТ Р 1.5---2001]
				\end{stdquote}
				
				\point В данном документе для визуального сходства с примерами из регламентирующего оформление стандартов ГОСТ~1.5---2001 в математическом шрифте использована подмена цифр, латиницы, букв греческого алфавита и некоторых других знаков на символы основного шрифта документа.
			
					\begin{equation}
						\forall x \in X, \quad \exists y \leq \epsilon
					\end{equation}
					\begin{equation}
						\alpha, A, \beta, B, \gamma, \Gamma, \pi, \Pi, \phi, \varphi, \Phi
					\end{equation}
					\begin{equation}
					\cos (2\theta) = \cos^2 \theta - \sin^2 \theta
					\end{equation}
					\begin{equation}
					\lim_{x \to \infty} \exp(-x) = 0
					\end{equation}
					\begin{equation}
					a \bmod b
					\end{equation}
					\begin{equation}
					x \equiv a \pmod b
					\end{equation}
					\begin{equation}
					k_{n+1} = n^2 + k_n^2 - k_{n-1}
					\end{equation}
					\begin{equation}
					f(n) = n^5 + 4n^2 + 2 |_{n=17}
					\end{equation}
					\begin{equation}
					A_{m,n} =
					\begin{pmatrix}
					a_{1,1} & a_{1,2} & \cdots & a_{1,n} \\
					a_{2,1} & a_{2,2} & \cdots & a_{2,n} \\
					\vdots  & \vdots  & \ddots & \vdots  \\
					a_{m,1} & a_{m,2} & \cdots & a_{m,n}
					\end{pmatrix}
					\end{equation}
					\begin{equation}
					\frac{\frac{1}{x}+\frac{1}{y}}{y-z}
					\end{equation}
					\begin{equation}
					  x = a_0 + \cfrac{1}{a_1
					  	+ \cfrac{1}{a_2
					  		+ \cfrac{1}{a_3 + \cfrac{1}{a_4} } } }
					\end{equation}
					\begin{equation}
					\frac{
						\begin{array}[b]{r}
						\left( x_1 x_2 \right)\\
						\times \left( x'_1 x'_2 \right)
						\end{array}
					}{
						\left( y_1y_2y_3y_4 \right)
					}
					\end{equation}
					\begin{equation}
					\sqrt[6]{\frac{a}{b}} \quad \int\limits_{\alpha}^{\omega} F_\text{инд.}
					\end{equation}
					\begin{equation}
					\sqrt[n]{1+x+x^2+x^3+\ldots}
					\end{equation}
					\begin{equation}
					\sum_{\substack{
							0<i<m \\
							0<j<n
						}}
						P(i,j)
					\end{equation}
					\begin{equation}
					( a ), [ b ], \{ c \}, | d |, \| e \|,
					\langle f \rangle, \lfloor g \rfloor,
					\lceil h \rceil, \ulcorner i \urcorner
					\end{equation}
					\begin{equation}
					P\left(A=2\middle|\frac{A^2}{B}>4\right)
					\end{equation}
					
					
			\subsection{Повторение положений другого стандарта}
			
				\point Регламентированы следующие требования к оформлению повторений положений других стандартов.
				
				\begin{stdquote}
					\par 4.8.4 В случае, когда в стандарте целесообразно повторить какое-либо положение (или его фрагмент) другого межгосударственного стандарта, это положение (фрагмент) заключают в рамки из тонких линий, а после него приводят в квадратных скобках ссылку на данный стандарт с указанием года его принятия (но без указания обозначения идентичного международного или регионального стандарта, если оно приведено согласно 8.10, или 8.11, или 8.12) и отделенных от обозначения стандарта запятой наименования и номера структурного элемента, в котором приведено это положение (фрагмент). Если повторяемое положение образует в стандарте отдельный структурный элемент (пункт, подпункт, терминологическую статью) или его целесообразно изложить как примечание, то номер этого структурного элемента или слово «Примечание --- » приводят вне рамки вверху слева.
					\par [ГОСТ 1.5---2001]
				\end{stdquote}
				
				\note{В данном классе текст повторяемых положений выводится более мелким шрифтом. Регламентирующий оформление таких элементов ГОСТ~1.5---2001 этого не требует, однако повторения положений других стандартов в нём и других стандартах оформлены именно так.}
				
				\point Для повторения положений другого стандарта необходимо использовать окружение \textbf{stdquote}.
				
			\subsection{Примечания}
			
				\point Регламентированы следующие требования к оформлению примечаний.
				
				\begin{stdquote}
					\par 4.9.2 Примечание печатают с прописной буквы и начинают с абзацного отступа. В конце текста примечания (вне зависимости от количества предложений в нем) ставят точку.
					\par Примечание помещают непосредственно после положения (графического материала), к которому относится это примечание.
					\par Примечание к таблице помещают в конце таблицы над линией, обозначающей окончание таблицы, как показано на рисунках 16-18. При этом примечание отделяют от таблицы сплошной тонкой горизонтальной линией.
					\par 4.9.3 Одно примечание не нумеруют, а после слова «Примечание» ставят тире.
					\par Несколько примечаний нумеруют по порядку арабскими цифрами. При этом после слова «Примечания» не ставят двоеточие.
					\par 4.9.4 Примечания выделяют в стандарте уменьшенным размером шрифта. Слово «Примечание» выделяют разрядкой.
					\par [ГОСТ 1.5---2001]
				\end{stdquote}
				
				\note{Пример одиночного примечния.}
				\begin{notes}
					\item Первое нумерованное примечание.
					\item Второе нумерованное примечание.
				\end{notes}
				
			\subsection{Сноски}
			
				\point Требования к оформлению сносок.
				
				\begin{stdquote}
					\par 4.10.1 Если необходимо пояснить отдельные слова, словосочетания или данные, приведенные в стандарте, то после них ставят надстрочный знак сноски.
					\par Сноску располагают в конце страницы, на которой приведено поясняемое слово (словосочетание или данные), а сноску, относящуюся к данным таблицы, - в конце таблицы над линией, обозначающей окончание таблицы. При этом сноску отделяют от текста короткой сплошной тонкой горизонтальной линией с левой стороны страницы, а от данных таблицы такой же линией, но проведенной до вертикальных линий, ограничивающих таблицу. Кроме этого, сноску выделяют уменьшенным размером шрифта.
					\par В конце сноски ставят точку.
					\par 4.10.2 Знак сноски ставят непосредственно после того слова (последнего слова словосочетания, числа, символа), к которому дается пояснение, а также перед поясняющим текстом.
					\par [ГОСТ 1.5---2001]
				\end{stdquote}
				
				\point\label{pnt:numfootnotes} Сноски могут нумероваться как цифрами, так и звёздочками.
				
				\begin{stdquote}
					\par 4.10.3 Знак сноски выполняют арабской цифрой со скобкой или в виде звездочки («*»), двух или трех звездочек («**» или «***»), помещая их на уровне верхнего обреза шрифта. Знак сноски отделяют от ее текста пробелом.
					\par 4.10.4 Для каждой страницы используют отдельную систему нумерации (обозначений) сносок. При этом применение более трех звездочек не допускается.
					\par [ГОСТ 1.5---2001]
				\end{stdquote}
				
				\point В данном классе по умолчанию используется обозначение сносок звёздочками\footnote{Для нумерации сносок цифрами используйте параметр класса \textbf{arbicfootnotes}.}.
					
				
		\OSTAppendix{(обязательное)}{Пакеты загружаемые классом}
		
			\point Класс загружает следующие пакеты:
				\begin{enumerate}
					\item \textbf{geometry}
					\item \textbf{indentfirst}
					\item \textbf{titlesec}
					\item \textbf{caption}
					\item \textbf{calc}
					\item \textbf{enumitem}
					\item \textbf{perpage}
				\end{enumerate}
				\subpoint Кроме того, загружаются и зависимые от них пакеты.
			
		\OSTAppendix{(рекомендуемое)}{Пример приложения}
		
			\section{Пример раздела в приложении}
			
				\subsection{Пример подраздела в приложении}
				
					\point Может быть это выглядит немного странно, но в пределах приложений так же могут быть и разделы, см. пункт \ref{pnt:appendixsections}.
		
					\point В этом приложении для примера нумерации приведены рисунок, таблица и формула. Они вне зависимости от параметров нумерации объектов нумеруются в пределах приложения.
					
						\subpoint В стандарте дано не достаточно точное ограничение вложенности структурных элементов. Вероятно этот пункт нарушает следующее требование.
						
						\begin{stdquote}
							\par 4.2.1.6 Количество номеров в нумерации структурных элементов стандарта не должно превышать четырех.
							\par [ГОСТ 1.5---2001]
						\end{stdquote}
					
					\begin{figure}[H]
						\centering
						\LARGE\TeX\quad\LaTeX\quad\LaTeXe\\
						\caption{Логотипы TeX, LaTeX и LaTeX2e}
					\end{figure}
					
					\begin{table}[H]
						\caption{Небольшая таблица}
						\begin{tabu}{|X[r]|X[c]|X[l]|}
							\hline
							Это    & небольшая & таблица \\ \hline
							В ней  & всего две & строки  \\ \hline
						\end{tabu}
					\end{table}
					
					\begin{equation}
						\lim_{x \to \infty} \exp(-x) = \int\limits_a^b \sqrt{\frac{A}{b + c^2}}
					\end{equation}
		
		\newpage		
		\begin{thebibliography}{9}
			\item ГОСТ Р 1.4---2004 Стандартизация в Российской Федерации. Стандарты организаций. Общие положения
			\item ГОСТ Р 1.5---2012 Стандартизация в Российской федерации. Стандарты национальные. Правила построения, изложения, оформления и обозначения
			\item ГОСТ 1.5---2001 Межгосударственная система стандартизации. Стандарты межгосударственные, правила и рекомендации по межгосударственной стандартизации. Общие требования к построению, изложению, оформлению, содержанию и обозначению
		\end{thebibliography}
	
	\end{OST}
	
	\noindent
	\begin{tabu}{@{}X[10,l]X[2,c]X[3,l]@{}}
		Руководитель разработки главный начальник  & \rule{\linewidth}{1pt} & Петров П.П.  \\
		Исполнитель рядовой сотрудник              & \rule{\linewidth}{1pt} & Сидоров С.С. \\
	\end{tabu}\vspace{1cm}
	
	\noindent СОГЛАСОВАНО:\vspace{.5cm}
	
	\noindent
	\begin{tabu}{@{}X[10,l]X[2,c]X[3,l]@{}}
		Кто-то очень важный    & \rule{\linewidth}{1pt} & Дмитриев Д.Д. \\
		Кто-то не менее важный & \rule{\linewidth}{1pt} & Фёдоров Ф.Ф.  \\
	\end{tabu}
	
\end{document}